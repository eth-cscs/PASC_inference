\documentclass{article} 

\usepackage[utf8]{inputenc}
\usepackage{amssymb,amsmath}

\usepackage{xcolor}
\usepackage[top=3cm, bottom=3cm, left=3cm, right=3cm, includefoot]{geometry}

\definecolor{dgreen}{rgb}{0.0, 0.5, 0.0} 

\usepackage{yfonts}

\begin{document}

\begin{displaymath}
 \theta^{*}(t) = \arg \min\limits_{\theta(t)} \int\limits_{0}^{T} g(x(t), \theta(t)) ~dt
  ~~ \approx ~~ 
  \arg \min\limits_{\theta(t)} \sum\limits_{t=0}^{T} g(x_t, \theta(t))
\end{displaymath}

\begin{displaymath}
 \textrm{error of the model on $[0,T]$}
\end{displaymath}

\begin{displaymath}
 \textrm{$\theta(t)$ is constant on clusters}
\end{displaymath}

\begin{displaymath}
 [\Theta^{*}, \Gamma^{*}(t) ] = \arg \min\limits_{\Theta,\Gamma} \int\limits_{0}^{T} \sum\limits_{k=1}^K \gamma_k(t) \cdot g(x(t), \theta_k) ~dt
  ~~ \approx ~~ 
  \arg \min\limits_{\Theta,\Gamma} \sum\limits_{t=0}^{T} \sum\limits_{k=1}^{K} \gamma_{k,t} \cdot g(x_t, \theta_k)
\end{displaymath}

\begin{displaymath}
 \theta_1, \theta_2, \dots, \theta_k ~\textrm{- model parameters on clusters}
\end{displaymath}

\begin{displaymath}
 \gamma_1(t), \gamma_2(t), \dots, \gamma_k(t) ~\textrm{- cluster indicator functions}
\end{displaymath}

\begin{displaymath}
 \theta_1 ~~~ \theta_2 ~~~ \theta_3 ~~~ \gamma_1(t) ~~~ \gamma_2(t) ~~~ \gamma_3(t) 
\end{displaymath}

\begin{displaymath}
 \textrm{continuous real-valued functions $\gamma_k(t) \in [0,1]$}
\end{displaymath}

\begin{displaymath}
 \textrm{smooth $\gamma_k(t)$ $~\Rightarrow~$ regularisation}
\end{displaymath}

\begin{displaymath}
 \gamma_k(t) \in \lbrace 0,1 \rbrace, ~~
 \gamma_k(t) =
 \left\lbrace
  \begin{array}{ll}
   1~ & \textrm{if $k$-th cluster is active in $t$} \\
   0~ & \textrm{if $k$-th cluster is inactive in $t$}
  \end{array}
 \right.
\end{displaymath}

\begin{displaymath}
 0 \leq \gamma_k(t) \leq 1, ~~
 \forall t: \sum\limits_{k=1}^K \gamma_k(t) = 1
\end{displaymath}

\begin{displaymath}
 \textrm{s.t.} ~~ 0 \leq \gamma_k(t) \leq 1, ~
 \forall t: \sum\limits_{k=1}^K \gamma_k(t) = 1
\end{displaymath}

\begin{displaymath}
 \int\limits_{0}^{T} \sum\limits_{k=1}^K \gamma_k(t) \cdot g(x(t), \theta_k) ~dt {\color{red} + \varepsilon^2 \sum\limits_{k=1}^K \int\limits_0^T (\partial_t \gamma_k(t))^2 dt}
  ~~ \approx ~~ 
  \sum\limits_{t=0}^{T} \sum\limits_{k=1}^{K} \gamma_{k,t} \cdot g(x_t, \theta_k)
    {\color{red} + \varepsilon^2 \sum\limits_{k=1}^K \sum\limits_{t=0}^{T-1} (\gamma_{k,t+1} - \gamma_{k,t})^2}
\end{displaymath}

\begin{displaymath}
 [\Theta^{*}, \Gamma^{*} ] = 
  \arg \min\limits_{\Theta,\Gamma} \sum\limits_{t=0}^{T} \sum\limits_{k=1}^{K} \gamma_{k,t} \cdot g(x_t, \theta_k)
  + \varepsilon^2 \sum\limits_{k=1}^K \sum\limits_{t=0}^{T-1} (\gamma_{k,t+1} - \gamma_{k,t})^2
\end{displaymath}

\begin{displaymath}
 \textrm{s.t.} ~~ 0 \leq \gamma_{k,t} \leq 1, ~
 \forall t: \sum\limits_{k=1}^K \gamma_{k,t} = 1
\end{displaymath}

\begin{displaymath}
 L(\Theta,\Gamma)
\end{displaymath}

\begin{center}
\begin{table}[h!]
\renewcommand{\tablename}{Algorithm}
\it
\begin{tabular}{p{0.95\linewidth}}
\hspace*{0.6CM}set feasible initial approximation $\Gamma_{0}$\\
\medskip
\hspace*{0.6CM}{\textbf{while}} $\Vert L(\Theta_{it},\Gamma_{it}) - L(\Theta_{it-1},\Gamma_{it-1}) \Vert > \varepsilon_L$\\[0.1cm]
\hspace*{1.2CM} solve $\Theta_{it} = \arg \min\limits_{\Theta} L(\Theta, \Gamma_{it})$ $~~~$ (with fixed $\Gamma_{it}$) \\[0.1cm]
\hspace*{1.2CM} solve $\Gamma_{it} = \arg \min\limits_{\Gamma} L(\Theta_{it}, \Gamma)$ $~~~$ (with fixed $\Theta_{it}$) \\[0.1cm]
\hspace*{1.2CM} $it = it + 1$ \\
\hspace*{0.6CM}{\textbf{endwhile}}
\end{tabular}
\end{table}
\end{center}

\begin{displaymath}
 \begin{array}{rcl}
  \theta_1 & = & [0,0,0] \\
  \theta_2 & = & [1,0,0] \\
  \theta_3 & = & [0,1,0] 
 \end{array}
\end{displaymath}


\begin{displaymath}
 \sum\limits_{t=0}^T \Vert x_t - \theta(t) \Vert^2 ~ \rightarrow ~ \min
\end{displaymath}



$y^t \in \{ 0 , 1\} $ : binary data 

\vspace{0.5cm} 

$P_X^t = \left(  P(x_1^t),...,P(x_n^t),1 \right)^T$ probability vector

\vspace{0.5cm} 

Log-likelihood at time $t$:
\begin{equation*}
\mathcal{L} (t) = y^t \log \left( \Lambda_X(t)^{T} P_X^t \right) + (1-y^t )\log \left( 1- \Lambda_X(t)^{T} P_X^t \right) 
\end{equation*}

\vspace{0.5cm}  


Want to solve the estimation problem
\begin{equation*}
\underset{ \Lambda_X(t): \ t \in [0,T]}{\max} \ \int_0^T \mathcal{L}(t) dt
\end{equation*}
such that:
\begin{equation*}
0 < \Lambda_X(t)^{T} P_X^t  < 1, \ \ \ \forall t
\end{equation*}

\begin{displaymath}
 Y^t = a + b Y^{t-1} + c X^{t-1} + \textrm{noise}
\end{displaymath}

\begin{displaymath}
 c \cong 0 ~~~ c \neq 0 ~~~ \Lambda \cong 0 ~~~ \Lambda = P[y^t \vert {\color{blue} x^t}] ~~~ \varepsilon = P[y^t ~\textrm{and NOT} ~ {\color{blue} x^t}]
 ~~~ y^t ~~~ {\color{blue} X} ~~~ {\color{blue} x^t} ~~~ Y
\end{displaymath}

\begin{displaymath}
 {\color{blue} X} ~~
 {\color{blue} \textrm{not}~X} ~~ Y ~~
 \lbrace y^t \rbrace, t = 0,\dots,T ~~
 {\color{blue}  \lbrace x^t \rbrace, t = 0,\dots,T} 
\end{displaymath}

\begin{displaymath}
 \begin{array}{rcl}
  P[y^t] & = & P[y^t ~\textrm{and}~ {\color{blue} x^t}] + P[y^t ~\textrm{and NOT} ~ {\color{blue} x^t}] \\
  & = & \Lambda P[{\color{blue} x^t}] + \varepsilon
 \end{array}
\end{displaymath}

\begin{displaymath}
\begin{array}{l}
\textrm{${\color{blue} X}$ has a causality impact on $Y$ if} \\
\textrm{for any $t$, event $y^t$ is happening if (and only if) event ${\color{blue} x^t}$ happened.}
\end{array}
\end{displaymath}

fit a stochastic model for Y and X, e.g. a linear VAR-X: \newline

If model with $c \cong 0$ is more statistically-significant than the one with $c \neq 0$

\begin{displaymath}
\textrm{$X$ has no \textbf{Granger-causality} impact on $Y$}
\end{displaymath}

Law of  the Total Probability: \newline

If $\Lambda \cong 0$ there is no causality impact from ${\color{blue} X}$ on $Y$. 

\begin{displaymath}
P[y^t] = 
\Lambda_1 {\color{red} (t)} P [{\color{blue} x_1^t}]
+
\Lambda_2 {\color{red} (t)} P [{\color{blue} x_2^t}]
+
\dots
+
\Lambda_n {\color{red} (t)} P [{\color{blue} x_n^t}]
\end{displaymath}
\begin{displaymath}
+ P[y^t ~\textrm{and not$({\color{blue} x_1^t}$ or $\dots$ or ${\color{blue} x_n^t})$} ]
\end{displaymath}
where $\Lambda_i {\color{red} (t)} = P [y^t \vert {\color{blue} x_i^t}]$ is possibly time-dependent due to unresolved scales. \newline

If $\Lambda_i \cong 0$ there is no causality impact from ${\color{blue} X_i}$ on $Y$. 


\end{document}

