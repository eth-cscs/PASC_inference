\documentclass[a4paper,10pt]{article}
\usepackage[onehalfspacing]{setspace}
\hoffset=-1.0cm\voffset=-2.0cm \textwidth=155mm\textheight=220mm

\usepackage{authblk}

\title{Causality Inference in a nonstationary and nonhomogenous framework}
\author[a]{Illia Horenko}
\author[a]{\underline{Patrick Gagliardini}}
\author[b]{William Sawyer}
\author[a]{\underline{Luk\'{a}\v{s} Posp\'{i}\v{s}il}}
\affil[a]{Universit\`{a} della Svizzera italiana, Lugano, Switzerland}
\affil[b]{Swiss National Supercomputing Centre (CSCS/ETH Zurich), Switzerland}
\affil[ ]{\textit {\{illia.horenko,patrick.gagliardini,lukas.pospisil\}@usi.ch, william.sawyer@cscs.ch}}

\date{}

\begin{document}
\pagenumbering{gobble}
  \maketitle



\par
\vspace{4mm}

\noindent
The project deploys statistical and computational techniques to develop a novel approach to causality inference 
in multivariate time series of economical data on equity and credit risks. The methods build on recent research of project participants, \cite{bib:HorenkoGerber}. They improve on classical approaches to causality analysis such as Granger causality (see \cite{bib:Granger})  by accommodating 
general forms of nonstationarity and non-homogenity resulting from unresolved and latent scale effects.  \newline
Emerging causality framework results in and is implemented through a clustering of time series based on a minimization of the averaged clustering
functional, which describes the mean distance between observation data
and its representation in terms of given number of abstract Bayesian causality models of a certain predefined class.
We are using finite element framework to the problem of time series analysis and propose a numerical scheme
for time series clustering \cite{bib:horenko}. \newline
One of the most challenging components of the emerging HPC implementation is a
Quadratic Programming (QP) problem with the combination of linear equality and bound inequality constraints.
In our contribution, we compare tree different QP algorithms to solve this problem -
Augmented Lagrangian method combined with active-set strategy \cite{bib:DosPos}, Interior-point methods \cite{bib:Boyd}, and
the modification of Spectral Projected Gradient method for QP \cite{bib:spg}, \cite{bib:Pos}.
We demonstrate and compare the efficiency of the methods solving practical benchmark problems. \newline





%% BIBLIOGRAPHY 
\begin{thebibliography}{}

\bibitem{bib:HorenkoGerber}
Horenko I., and Gerber S.:
{\em On inference of causality for discrete state models in a multiscale context}.
Proc. Natl. Acad. Sci. USA (PNAS), DOI: 10.1073/pnas.1410404111, (2014).

\bibitem{bib:Granger}
Granger C.:
{\em Investigating Causal Relations by Econometric Models and Cross-Spectral methods}.
Econometrica, 37, pp. 424--438, (1969).


\bibitem{bib:horenko} 
Horenko I.:
{\em Finite element approach to clustering of multidimensional time series}.
SIAM Journal of Scientific Computing 32, pp. 62--83, (2010).

\bibitem{bib:DosPos}
 Dost\'{a}l Z., Posp\'{i}\v{s}il L.:
 {\em Optimal iterative QP and QPQC algorithms}.
 Annals of Operations Research, DOI: 10.1007/s10479-013-1479-0, (2013).  

\bibitem{bib:Boyd}
  Boyd S., Vandenberghe L.:
  {\em Convex Optimization}. 
  Cambridge University Press, New York, 1st edition, (2004).

\bibitem{bib:spg} 
  Birgin E.G., Mart\'{i}nez J.M., and Raydan M.:
  {\em Nonmonotone spectral projected gradient methods on convex sets}.
  SIAM Journal on Optimization 10, pp. 1196--1211, (2000).

\bibitem{bib:Pos}
 Posp\'{i}\v{s}il L.:
 {\em Development of Algorithms for Solving Minimizing Problems with Convex Quadratic Function on Special Convex Sets and Applications}.
 PhD Thesis, V\v{S}B-TU Ostrava, (2015).


  

   
\end{thebibliography}


\end{document}
