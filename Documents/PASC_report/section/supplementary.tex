\subsection{Principle of proof by contradiction}

Let us suppose, that we want to (have to) prove the implication
\begin{equation}
 \label{eq:contradiction_implication}
 v_1 \Rightarrow v_2,
\end{equation}
where $v_1, v_2$ are statements\footnote{statement is a meaningful declarative sentence that is either true or false}. Suppose that "standart" direct proof in form
\begin{displaymath}
 v_1 \Rightarrow ~~~ 
 \hat{v}_1 \Rightarrow
 \dots
 \Rightarrow \hat{v}_m
 ~~~ \Rightarrow v_2
\end{displaymath}
is not suitable or too complicated. Here $\hat{v}_1, \dots, \hat{v}_m$ denote auxiliary statements presenting the small partial steps during the proof.

At first, let us recall the \emph{Truth table} of the implication and other interesting relationship between statements (i.e. composed statements). 
In Table \ref{table:statements}, $1$ denotes that the statement is \emph{true}, $0$ denotes \emph{false}, and $v'$ is the negation of statement $v$.

\begin{table}[h]
%\def\arraystretch{1.3}
\centering
\begin{tabular}{C{0.05\linewidth} | C{0.05\linewidth} || C{0.1\linewidth} | C{0.1\linewidth} | C{0.1\linewidth} }
\hline
 $v_1$ & $v_2$ & $v_1 \Rightarrow v_2$ & $v_1' \vee v_2$ & $v_1 \wedge v_2'$ \\
\hline
 1 & 1 & 1 & 1 & 0 \\
 1 & 0 & 0 & 0 & 1 \\
 0 & 1 & 1 & 1 & 0 \\
 0 & 0 & 1 & 1 & 0 \\
\hline
\end{tabular}
\caption{Truth of selected composed statements; notice that 3th and 4th columns are equivalent, 5th is the negation of 3th and 4th column}
\label{table:statements}
\end{table}

Please, notice that from the table it is clear that statement $v_1 \Rightarrow v_2$ is equivalent to $v_1' \vee v_2$. However, instead of proving this statement (the first one
or the second one, it does not matter which one, since they are equivalent), we can prove that the negation  of this statement is not true
\footnote{because if the negation is not true, than the original statement is true}. The negation of implication is given by $v_1 \wedge v_2'$, see Table \ref{table:statements}.
To see the real relationship between $v_1' \vee v_2$ and $v_1 \wedge v_2'$, please, see also so-called \emph{De-Morgan's laws}.

Therefore, if we want to prove the implication \eqref{eq:contradiction_implication}, then we rather examine $v_1 \wedge v_2'$, i.e. we suppose, that the assumptions of the implication are true and the result in not true in the same time.
To prove that this statement is not true, it is enought to show that it does not hold in any case, i.e. we state the \emph{contradiction}.

Moreover, if we consider additional quantifiers (like \emph{for all} or \emph{there exists}) and our statements depend on the parameters, 
then during the negation of original implication statement, we have to perform negation also to all quantifiers.
For example the negation of
\begin{displaymath}
 \forall \alpha: v_1(\alpha) \Rightarrow v_2(\alpha)
\end{displaymath}
is given by
\begin{displaymath}
 \exists \alpha: v_1(\alpha) \wedge v_2'(\alpha).
\end{displaymath}
So, if we prove that this $\alpha$ does not exist, then we are done.
